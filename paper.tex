\documentclass[conference]{IEEEtran}
\usepackage{amsmath,amssymb,amsfonts}
\usepackage{url}
\usepackage{subfigure}
\usepackage{booktabs,threeparttable,multirow}

% new math operators
\DeclareMathOperator{\abs}{abs}

% todo command
\usepackage{marginnote}
\newcounter{todocnt}
\newcommand{\Sim}{\textsc{Simon}} 
\setcounter{todocnt}{0}
\newcommand{\todo}[1]{\stepcounter{todocnt}{\tt {[#1]}} \marginpar{{$\blacksquare$ \thetodocnt}}}  
\newcommand{\specialcell}[2][c]{%
  \begin{tabular}[#1]{@{}c@{}}#2\end{tabular}}

\hyphenation{op-tical net-works semi-conduc-tor}
\IEEEoverridecommandlockouts
\begin{document}

\title{Countermeasures against co-location attempts in the cloud}


\author{\IEEEauthorblockN{Dolan Murvihill, Nathan Wells
}
\IEEEauthorblockA{Worcester Polytechnic Institute, 
Worcester, MA 01609, USA\\
Email: \texttt{\{dm, nhwells\}@wpi.edu}
}}
\maketitle

\begin{abstract}
Recent research has demonstrated that cloud services such as Amazon AWS do not
provide perfect isolation --- that there are side channels between virtual
machines that reside on the same host. In particular, it is possible, under
some circumstances, to use cache timing information to perform a full AES key
recovery against a VM sharing the same cache.

Other work has shown that cloud services such as Amazon AWS provision virtual
machines using straightforward, often predictable algorithms, and others have
demonstrated that it is possible to achieve the required co-location to perform
a co-location attack. We would like to interrupt this step in the kill chain,
by developing a provisioning strategy that makes co-location much more
difficult to force or predict, without sacrificing too much performance.
\end{abstract}

\section{Motivation}
\todo{Know what you want to do and why that is interesting (maybe with bullet points). But do not write this section until you know what you actually have done so that the motivation fits your work.}

To generate this pdf, you need a latex implementation. I recommend Texniccenter in Windows, but there are others. For your references (the .bib file) jabref is the only helpful editor I have used.
Refer to sections like this: Section~\ref{sec:background} and to references like this: Ristenpart et al.~\cite{Ristenpart_hey}. \\{\tiny There is also a word template available.}


We will stick to the following timeline:

\begin{itemize}
	\item 2/22: Project goals and outline defined
	\item 3/1 : Related work identified and described in report.
	\item 3/15: At least 1/3 of your anticipated work should be completed and documented
	\item 3/22: At least 2/3 of your anticipated work should be completed and documented
	\item 3/29: All of your anticipated work should be completed and documented
	\item 4/5:  Your results and outcomes are now also documented and included in the paper
	\item 4/12: Final submission of of complete paper for review
	\item 4/21: Reviews complete
	\item 4/26: Submission of final version addressing reviewers comments.
	\item 5/4 : Presentation of your project in class.
\end{itemize}

\section{Background}\label{sec:background}
\todo{You should find and describe related work early on. Know what other people have done.}

\section{Work Description}
Here you describe the work you have performed, problems you have solved and methods you have used. There is a fine balance between brevity and conciseness and ensuring that other people, if investing the time, would be able to reproduce your results given this description.



\section{Results}
\todo{here you will present and discuss your outcomes: implementation results or measurements or other project outcomes}

\section{Conclusion}
\todo{TBD last}



%\bibliographystyle{IEEEtran}

\end{document}
