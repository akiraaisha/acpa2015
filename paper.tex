\documentclass[conference]{IEEEtran}
\usepackage{amsmath,amssymb,amsfonts}
\usepackage{url}
\usepackage{subfigure}
\usepackage{booktabs,threeparttable,multirow}

% new math operators
\DeclareMathOperator{\abs}{abs}

% todo command
\usepackage{marginnote}
\newcounter{todocnt}
\newcommand{\Sim}{\textsc{Simon}} 
\setcounter{todocnt}{0}
\newcommand{\todo}[1]{\stepcounter{todocnt}{\tt {[#1]}} \marginpar{{$\blacksquare$ \thetodocnt}}}  
\newcommand{\specialcell}[2][c]{%
  \begin{tabular}[#1]{@{}c@{}}#2\end{tabular}}

\hyphenation{op-tical net-works semi-con-duc-tor}
\IEEEoverridecommandlockouts
\begin{document}

\title{Countermeasures against co-location attempts in the cloud}


\author{\IEEEauthorblockN{Dolan Murvihill, Nathan Wells
}
\IEEEauthorblockA{Worcester Polytechnic Institute, 
Worcester, MA 01609, USA\\
Email: \texttt{\{dm, nhwells\}@wpi.edu}
}}
\maketitle

\begin{abstract}
Recent research has demonstrated that cloud services such as Amazon AWS do not provide perfect isolation~--- that there
  are side channels between virtual machines that reside on the same host.
In particular, it is possible, under some circumstances, to use cache timing information to perform a full AES key
  recovery against a VM sharing the same cache.

Other work has shown that cloud services such as Amazon AWS provision virtual machines using straightforward, often
  predictable algorithms, and others have demonstrated that it is possible to achieve the required co-location to
  perform a co-location attack.
We would like to discern whether Amazon AWS has implemented any countermeasures against co-location attacks since the
first successful cloud co-location six years ago; we will repeat their procedure in an attempt to reproduce their
results. If time permits, we will try to improve on their attack.
\end{abstract}

\section{Motivation}
\todo{Know what you want to do and why that is interesting (maybe with bullet points). But do not write this section until you know what you actually have done so that the motivation fits your work.}



\begin{itemize}
  \item Amazon's EC2 service is very popular and used for by variety of customers, some of which have sensitive data.
  \item This makes VMs on EC2 a worthwhile target.
  \item Ristenpart et al. have demonstrated that it is possible to create a VM co-resident with a target instance and
    that existing side-channel attacks can be employed against co-resident instances on the cloud.
  \item We expect that Amazon has since adjusted their VM location algorithms to make the Ristenpart attack more
    difficult. We would like to confirm that these changes have been made, by repeating the Ristenpart attack.
  \item If time permits, we will attempt to defeat Amazon's new countermeasures, or improve on the reliability of the
    Ristenpart attack, whatever is more appropriate.
\end{itemize}

\section{Background}\label{sec:background}
\todo{You should find and describe related work early on. Know what other people have done.}

\section{Work Description}
Our planned work consists of two steps: a ``network cartography'' phase which consists of repeating key measurements
  from the Ristenpart paper, and an ``attack'' phase in which we try to maximize the probability of locating a new
  Amazon EC2 virtual machine on the same physical host as a target.
In an effort to accelerate the project, we will contact the authors of the original study and ask to use their source
  code.
\subsection{Cartography Phase}
Our goal for this step is to find the relationship between AWS instance type and IP address.
We will use the EC2 API to create a large number of virtual machines, record their IP addresses, and then terminate
  them.
We will then look for patterns between IP address allocation and instance type.
We expect to complete this task by March 22.
\subsection{Attack Phase}
Following the methods of the Ristenpart paper, we will create a simplistic program to transfer information over a
cross-VM covert channel between two VMs under our control.
We will assume that if this method succeeds, the VMs are co-located.
We will then try to correlate the success of this method with other relationships between the VMs, such as numerically
  close IP addresses, or identical IP addresses for the VMs' ``domain zero'' host controller.
We will then use those other relationships to test co-location with arbitrary VMs.

We will first implement a brute force attack that creates a large number of virtual machines with the same parameters,
  and measure their co-location rate with a number of target VMs.
We will then implement the Ristenpart et al. improvement that creates the attack virtual machines at approximately the
  same time as the target VMs, and measure the co-location rate similarly.
If time permits, we will implement other improvements as well.

We expect to finish this task by March 29.

\section{Results}
\todo{here you will present and discuss your outcomes: implementation results or measurements or other project outcomes}

\section{Conclusion}
\todo{TBD last}



%\bibliographystyle{IEEEtran}

\end{document}
